\documentclass[11pt,letterpaper]{article}

\addtolength{\oddsidemargin}{-.875in}
\addtolength{\evensidemargin}{-.875in}
\addtolength{\textwidth}{1.75in}

\addtolength{\topmargin}{-.875in}
\addtolength{\textheight}{1.75in}

\usepackage[utf8]{inputenc}
\usepackage{caption} % for table captions
\usepackage{amsmath} % for multi-line equations and piecewises
\DeclareMathOperator{\sign}{sign}
\usepackage{graphicx}
\usepackage{relsize}
\usepackage{xspace}
\usepackage{verbatim} % for block comments
\usepackage{subcaption} % for subfigures
\usepackage{enumitem} % for a) b) c) lists
\newcommand{\Cyclus}{\textsc{Cyclus}\xspace}%
\newcommand{\Cycamore}{\textsc{Cycamore}\xspace}%
\newcommand{\deploy}{\texttt{d3ploy}\xspace}%
\newcommand{\Deploy}{\texttt{D3ploy}\xspace}%
\usepackage{tabularx}
\usepackage{color}
\usepackage{multirow}
\usepackage{float} 
\usepackage[acronym,toc]{glossaries}
%\include{acros}
\definecolor{bg}{rgb}{0.95,0.95,0.95}
\newcolumntype{b}{X}
\newcolumntype{f}{>{\hsize=.15\hsize}X}
\newcolumntype{s}{>{\hsize=.5\hsize}X}
\newcolumntype{m}{>{\hsize=.75\hsize}X}
\newcolumntype{r}{>{\hsize=1.1\hsize}X}
\usepackage{titling}
\usepackage[hang,flushmargin]{footmisc}
\renewcommand*\footnoterule{}
\usepackage{tikz}

\usetikzlibrary{shapes.geometric,arrows}
\tikzstyle{process} = [rectangle, rounded corners, 
minimum width=1cm, minimum height=1cm,text centered, draw=black, 
fill=blue!30]
\tikzstyle{arrow} = [thick,->,>=stealth]

\graphicspath{}

\begin{document}

\section{Advection}

	\subsection{advect-1}

	\begin{itemize}
		\item \textit{advect-1.i}
		\item 1D generated mesh with libmesh
		\item Uses DG Kernels
		\item InflowBC and OutflowBC
		\item Transient problem
	\end{itemize}

    Figure \ref{fig:advect-1} shows the results.
    It seems like the variable has to be a CONSTANT MONOMIAL.

	\begin{figure}[htbp!]
		\centering
		\begin{subfigure}[t]{0.4\textwidth}
			\centering
			\includegraphics[width=\linewidth]{advect-1A}
			\caption{t=0.}
		\end{subfigure}
		\begin{subfigure}[t]{0.4\textwidth}
			\centering
			\includegraphics[width=\linewidth]{advect-1B}
			\caption{t=100.}
		\end{subfigure}
		\hfill
		\caption{Advected density.}
		\label{fig:advect-1}
	\end{figure}

	\subsection{periodic\_bc2}

	\begin{itemize}
		\item \textit{moose/examples/ex04\_bcs/periodic\_bc2.i}
		\item 1D generated mesh with libmesh
		\item Periodic BCs
		\item Transient problem
	\end{itemize}

    In \textit{advect-1-bc.i} I tried to add periodicBCs to the previous problem and it does not work.
    Here I tried to isolate the problem.
    Figure \ref{fig:periodic} shows the results.
    It does not work if the valiable is a CONSTANT MONOMIAL.
    It works if the variable is FIRST order (either MONOMIAL or LAGRANGE).

	\begin{figure}[htbp!]
		\centering
		\begin{subfigure}[t]{0.4\textwidth}
			\centering
			\includegraphics[width=\linewidth]{periodic_bc2_1}
			\caption{t=0.}
		\end{subfigure}
		\begin{subfigure}[t]{0.4\textwidth}
			\centering
			\includegraphics[width=\linewidth]{periodic_bc2_2}
			\caption{t=20.}
		\end{subfigure}
		\hfill
		\caption{Periodic BCs.}
		\label{fig:periodic}
	\end{figure}

	\subsection{advecss-1}

	\begin{itemize}
		\item \textit{advecss-1.i}
		\item 1D generated mesh with libmesh
		\item Uses DG Kernels
		\item Inflow and OutflowBC
		\item Steady problem
	\end{itemize}

    Same as previos problem but steady state and adds a source.
    Figure \ref{fig:advecss-1} shows the results.

	\begin{figure}[htbp!]
		\centering
		\includegraphics[height=5cm]{advec1-ss}
		\caption{Steady state solution.}
		\label{fig:advecss-1}
	\end{figure}

	\subsection{advect-2}

	\begin{itemize}
		\item \textit{advect-2.i}
		\item 1D generated mesh with libmesh
		\item Uses DG Kernels
		\item TemperatureInflowBC and TemperatureOutflowBC
		\item Transient problem
	\end{itemize}

    Very similar to \textit{advect-1.i}
    Advects BC.
    Figure \ref{fig:advect-2} shows the results.

	\begin{figure}[htbp!]
		\centering
		\includegraphics[height=5cm]{advec2}
		\caption{Advects BC.}
		\label{fig:advect-2}
	\end{figure}

	\subsection{advect-3}

	\begin{itemize}
		\item \textit{advect-3.i}
		\item 1D generated mesh with libmesh
		\item Uses DG Kernels
		\item TemperatureInflowBC and TemperatureOutflowBC
		\item Transient problem
	\end{itemize}

    Similar to \textit{advect-1.i}
    Adds a point source and solves for temperature.
    Figure \ref{fig:advect-3} shows the results.

	\begin{figure}[htbp!]
		\centering
		\includegraphics[height=5cm]{advect-3}
		\caption{Advected temperature from point source.}
		\label{fig:advect-3}
	\end{figure}

	\subsection{advect-4}

	\begin{itemize}
		\item \textit{advect-4.i}
		\item pseudo-1D: 2D-coolant.msh
		\item Uses DG Kernels
		\item TemperatureInflowBC and TemperatureOutflowBC
		\item Transient problem
	\end{itemize}

    Similar to \textit{advect-3.i} but has a $q''$ on the wall.
    Figure \ref{fig:advect-4} shows the results.

	\begin{figure}[htbp!]
		\centering
		\includegraphics[height=5cm]{advec4}
		\caption{Advects temperature while wall is been heated.}
		\label{fig:advect-4}
	\end{figure}

\pagebreak
\bibliographystyle{plain}
% \bibliography{bibliography}

\end{document}

	% \begin{figure}[htbp!]
	% 	\centering
	% 	\includegraphics[height=5cm]{1D-fuel-eig}
	% 	\caption{Steady state Group 1 and 2 fluxes.}
	% 	\label{fig:1D-fuel-eig}
	% \end{figure}