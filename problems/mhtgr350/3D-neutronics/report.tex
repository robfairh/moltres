\documentclass[11pt,letterpaper]{article}

\addtolength{\oddsidemargin}{-.875in}
\addtolength{\evensidemargin}{-.875in}
\addtolength{\textwidth}{1.75in}

\addtolength{\topmargin}{-.875in}
\addtolength{\textheight}{1.75in}

\usepackage[utf8]{inputenc}
\usepackage{caption} % for table captions
\usepackage{amsmath} % for multi-line equations and piecewises
\DeclareMathOperator{\sign}{sign}
\usepackage{graphicx}
\usepackage{relsize}
\usepackage{xspace}
\usepackage{verbatim} % for block comments
\usepackage{subcaption} % for subfigures
\usepackage{enumitem} % for a) b) c) lists
\newcommand{\Cyclus}{\textsc{Cyclus}\xspace}%
\newcommand{\Cycamore}{\textsc{Cycamore}\xspace}%
\newcommand{\deploy}{\texttt{d3ploy}\xspace}%
\newcommand{\Deploy}{\texttt{D3ploy}\xspace}%
\usepackage{tabularx}
\usepackage{color}
\usepackage{multirow}
\usepackage{float} 
\usepackage[acronym,toc]{glossaries}
%\include{acros}
\definecolor{bg}{rgb}{0.95,0.95,0.95}
\newcolumntype{b}{X}
\newcolumntype{f}{>{\hsize=.15\hsize}X}
\newcolumntype{s}{>{\hsize=.5\hsize}X}
\newcolumntype{m}{>{\hsize=.75\hsize}X}
\newcolumntype{r}{>{\hsize=1.1\hsize}X}
\usepackage{titling}
\usepackage[hang,flushmargin]{footmisc}
\renewcommand*\footnoterule{}
\usepackage{tikz}

\usetikzlibrary{shapes.geometric,arrows}
\tikzstyle{process} = [rectangle, rounded corners, 
minimum width=1cm, minimum height=1cm,text centered, draw=black, 
fill=blue!30]
\tikzstyle{arrow} = [thick,->,>=stealth]

\graphicspath{}

\begin{document}

\section{3D-neutronics}

\subsection{3D-assembly-30-homo-eig}

	\begin{itemize}
		\item Mesh: \textit{3D-assembly-30deg-reflec.msh}
		\item Eigenvalue problem.
	\end{itemize}

Figure \ref{fig:3D-assembly-eig} displays the geometry.
Figure \ref{fig:3D-assembly-eig1} shows the results.
Figure \ref{fig:3D-assembly-eig2} shows the eigenvalue vs number of iterations.

	\begin{figure}[htbp!]
		\centering
		\includegraphics[height=5cm]{3D-assembly-30deg-reflec-meshB}
		\caption{\textit{3D-assembly-30deg-reflec} scaled down geometry.}
		\label{fig:3D-assembly-eig}
	\end{figure}

	\begin{figure}[htbp!]
		\centering
		\includegraphics[height=8cm]{3D-assembly-30-homo-eig_axial}
		\caption{Group 1 and 2 steady-state axial flux.}
		\label{fig:3D-assembly-eig1}
	\end{figure}

	\begin{figure}[htbp!]
		\centering
		\includegraphics[height=8cm]{3D-assembly-eig}
		\caption{Eigenvalue calculation convergence.}
		\label{fig:3D-assembly-eig2}
	\end{figure}

\subsection{3D-assembly-homo-action}

	\begin{itemize}
		\item Mesh: \textit{3D-assembly-30deg-reflec.msh}
		\item Transient problem.
		\item Fuel, Moderator, and coolant are homogenized.
	\end{itemize}

Figure \ref{fig:3D-assembly-eig} displays the geometry.
Figure \ref{fig:3D-assembly-homo1} shows the results.

	\begin{figure}[htbp!]
		\centering
		\includegraphics[height=8cm]{3D-assembly-homo}
		\caption{Group 1 and 2 axial flux at 1 msec.}
		\label{fig:3D-assembly-homo1}
	\end{figure}

\subsection{3D-fullcore-60-homo-eig}

	\begin{itemize}
		\item Mesh: \textit{3D-fullcore-60-homo.msh}
		\item Eigenvalue problem.
	\end{itemize}

Figure \ref{fig:3D-fullcore-60-homo} displays the geometry.
Figure \ref{fig:3D-fullcore-60-homo1} shows the results.

	\begin{figure}[htbp!]
		\centering
		\begin{subfigure}[t]{0.4\textwidth}
			\centering
		\includegraphics[height=5cm]{3D-fullcore-60-homo-meshA2}
			\caption{XZ-plane.}
		\end{subfigure}
		\begin{subfigure}[t]{0.4\textwidth}
			\centering
		\includegraphics[height=5cm]{3D-fullcore-60-homo-meshB2}
			\caption{XY-plane.}
		\end{subfigure}
		\hfill
		\caption{\textit{3D-fullcore-60-homo} geometry.}
		\label{fig:3D-fullcore-60-homo}
	\end{figure}

	\begin{figure}[htbp!]
		\centering
		\begin{subfigure}[t]{0.4\textwidth}
			\centering
			\includegraphics[width=\linewidth]{3D-fullcore-60-radial1}
			\caption{Radial flux between points (0,0,400) and (259,150,400).}
		\end{subfigure}
		\begin{subfigure}[t]{0.4\textwidth}
			\centering
			\includegraphics[width=\linewidth]{3D-fullcore-60-axial1}
			\caption{Axial flux between points (85,55,0) and (85,55,1073).}
		\end{subfigure}
		\hfill
		\caption{Group 1 and 2 steady state axial fluxes. K$_{eff}$ = 1.430523.}
		\label{fig:3D-fullcore-60-homo1}
	\end{figure}

\pagebreak
\bibliographystyle{plain}
% \bibliography{bibliography}

\end{document}
